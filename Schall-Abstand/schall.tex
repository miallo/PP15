\subsubsection{Abstandsgesetz und Intensität}

%Motivation:
Wenn Reflektion und die Mitbewegung des Mediums (Wasser) vernachlässigt werden, sollten sich die Abhängigkeiten bei der Ausbreitung des Schalls vom Piezokristall, der Schallwellen mit Frequenzen im Ultraschallbereich abgibt
ähnlich verhalten, wie die Gesetze einer punktförmigen Schallquelle. 

%Quelle Tipler S.589

Der Grund dafür ist, dass die genutzen Entfernungen zwischen Sender und Empfänger vielfach größer sind als die Größe des schwingenden Kristalls. 
In einem homogenem Medium verteilt sich dann die ausgesendete Energie der Welle in drei Dimensionen gleichmäßig im Abstand $r$ über eine Kugelfläche $A$. Wird nun mit $<P>$ die mittlere vom Sender abgestrahlte Leistung 
bezeichnet, dann ist also zu erwarten, dass die Intensität des Signals sich antiproportional zu $r^2$ verhält:

\begin{align}
 I \propto \frac{<P>}{r^2}
\end{align}
Wenn nun $v$ die Ausbreitungsgeschwindigkeit im homogenen Medium und $<w>$ die mittlere Energiedichte bezeichnet, dann kann die mittlere Leistung kann auch geschrieben werden als:
\begin{align}
 <P>  = <w> A v
\end{align}
Woraus sich direkt für die Intensität ergibt:
\begin{align}
 I = <w> v
\end{align}
Die nun noch unbekannte Energiedichte von Schallwellen kann man aus der Betrachtung einer harmonischen Welle gewinnen. Dabei ergibt sich ganz allgemein, dass diese 
proportional ist zum Quadrat der Amplitude $A$ und der Kreisfrequenz $\omega$:
\begin{align}
 <\omega> = \frac{1}{2} \rho \omega^2 A^2
\end{align}
Mit $\rho$ wurde hier wie üblich die Dichte des Mediums bezeichnet.
Insgesamt ergibt sich also 
\begin{align}
 I = \frac{1}{2} \rho \omega^2 A^2 = \frac{1}{2} \frac{P_{max}^2}{\rho v}  
\end{align}
Es wurde verwendet, dass man speziell für Schallwellen zeigen kann, dass der Zusammenhang zwischen der maximalen Druckamplitude $P_{max}$ und der Amplitude der Schallwelle gegeben ist durch:
\begin{align}
 P_{max} = \rho \omega v A
\end{align}
Als Ergebnis lässt sich festhalten, dass die Intensität der Schallwelle proportional ist zum Quadrat der Druckamplitude.

\subsubsection{Schallgeschwindigkeit}
Für Schallwellen kann man in Fluiden, wie Wasser oder Luft folgende Gleichung für die Schallgeschwindigkeit finden:
\begin{align}
 v = \sqrt{  \frac{K}{\rho}  }
\end{align}
Wie oben bezeichnet $\rho$ die Massendichte. Mit $K$ wurde hier das sog. Kompressiblitätsmodul bezeichnet. Dieses ist definiert als das Verhältnis der relativen Druck und Volumenänderungen:
\begin{align}
 K =  \frac{\Delta P}{\Delta V / V }
\end{align}
Es wird hierdurch ersichtlich, dass die Schallgeschwindigkeit also ganz wesentlich vom Medium abhängt. 
Sie beträgt beispielsweise für Wasser je nach Salzgehalt und Temperatur circa 1500 m/s  und für Luft 340 m/s.

